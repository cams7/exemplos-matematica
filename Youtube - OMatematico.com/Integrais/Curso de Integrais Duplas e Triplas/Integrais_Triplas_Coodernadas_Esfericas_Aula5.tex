\begin{enumerate}
	\item Exercício retirado da página 899 de \cite{Jon_Rogawski_calculo_v2}
	
	Encontre a massa de uma esfera S de raio 4 centrada na origem com densidade de massa dado abaixo.
	
	\begin{equation*}
		\mu = f(x,y,z) = x^2 + y^2
	\end{equation*}
	
	\begin{align*}
		\mu = \dfrac{m}{v} \Rightarrow m = \mu v \Rightarrow \int dm = \int \mu dv \Rightarrow\\ m = \int \mu dv = \iiint_R \mu\, dv = \iiint_S \left(x^2 + y^2\right)\, dv
	\end{align*}
	\begin{equation*}
		S = \{(x,y,z) \,|\, x^2 + y^2 + z^2 \leq 16\}
	\end{equation*}
	\begin{align*}
		x^2 + y^2 = (r\sen(\varphi)\cos(\theta))^2 + (r\sen(\varphi)\sen(\theta))^2 =\\ r^2\sen^2(\varphi)\cos^2(\theta) + r^2\sen^2(\varphi)\sen^2(\theta) = r^2\sen^2(\varphi)\left(\cos^2(\theta) + \sen^2(\theta)\right) =\\ r^2\sen^2(\varphi)
	\end{align*}
	\begin{equation*}
		dv = dxdydz = r^2\sen(\varphi)\, dr d\theta d\varphi
	\end{equation*}
	\begin{equation*}
		0 \leq r \leq 4,\, 0 \leq \theta \leq 2\pi,\, 0 \leq \varphi \leq \pi
	\end{equation*}
	\begin{align*}
		m = \iiint_S \left(x^2 + y^2\right)\, dv = \int_0^4 \int_0^{2\pi} \int_0^{\pi} \left(r^2\sen^2(\varphi)\right)r^2\sen(\varphi)\, d\varphi d\theta dr =\\ \int_0^4 \int_0^{2\pi} \int_0^{\pi} r^4\sen^3(\varphi)\, d\varphi d\theta dr = \int_0^4 r^4\,dr \int_0^{2\pi} d\theta \int_0^{\pi} \sen^2\sen(\varphi)\, d\varphi =\\ \left[\dfrac{r^5}{5}\right]_0^4 \left[\theta\right]_0^{2\pi} \int_0^{\pi} \left(1 - \cos^2(\varphi)\right)\sen(\varphi)\, d\varphi =\\ \dfrac{1024}{5}2\pi \left(\int_0^{\pi} \sen(\varphi)\, d\varphi - \int_0^{\pi} \cos^2(\varphi)\sen(\varphi)\, d\varphi\right) =\\ \dfrac{2048\pi}{5} \left(\int_0^{\pi} \sen(\varphi)\, d\varphi + \int_0^{\pi} u^2\, du\right) = \dfrac{2048\pi}{5} \left[- \cos(\varphi)+ \dfrac{u^3}{3}\right]_0^{\pi} =\\ \dfrac{2048\pi}{5} \left[\dfrac{- 3\cos(\varphi) + \cos^3(\varphi)}{3}\right]_0^{\pi} = \dfrac{2048\pi}{15} \left[-\cos(\varphi)\left(3 - \cos^2(\varphi)\right)\right]_0^{\pi} =\\ \dfrac{2048\pi}{15} \left[-\cos(\pi)\left(3 - \cos^2(\pi)\right) + \cos(0)\left(3 - \cos^2(0)\right)\right] = \dfrac{2048\pi}{15} \left[\left(3 - 1\right) + \left(3 - 1\right)\right] =\\ \dfrac{2048\pi}{15} \left(2 + 2\right) = \dfrac{8192\pi}{15}
	\end{align*}
	\begin{equation*}
		u = \cos(\varphi) \Rightarrow -du = \sen(\varphi)\,d\varphi
	\end{equation*}
\end{enumerate}