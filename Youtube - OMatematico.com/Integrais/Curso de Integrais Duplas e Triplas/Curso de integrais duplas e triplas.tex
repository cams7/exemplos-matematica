%Preâmbulo
\documentclass[12pt, a4paper]{article}
\usepackage[brazilian]{babel}
\usepackage[utf8]{inputenc}
%\usepackage[left=2cm,right=2cm,top=2cm,bottom=2cm]{geometry}
\usepackage{indentfirst}

\usepackage{amsmath, amsfonts, amssymb}
\DeclareMathOperator{\sen}{sen}
\DeclareMathOperator{\tg}{tg}
\DeclareMathOperator{\cossec}{cossec}
\DeclareMathOperator{\arcsen}{arcsen}
\DeclareMathOperator{\arctg}{arctg}
\DeclareMathOperator{\arccossec}{arccossec}
\newcommand{\limite}{\displaystyle\lim}
\newcommand{\integral}{\displaystyle\int}
\usepackage[normalem]{ulem}
\newcommand{\overstrike}[1]{\ifmmode\text{\sout{\ensuremath{#1}}}\else\sout{#1}\fi}

\usepackage{graphicx}
\graphicspath{{img/}}
\usepackage{float}

\title{Curso de integrais duplas e triplas}
\author{César Antônio de Magalhães \\ ceanma@gmail.com}
\date{\today}

%Corpo do texto
\begin{document}
	\maketitle\newpage
	
	\tableofcontents\newpage
	
	%\listoftables\newpage
	
	\listoffigures\newpage
	
	%\begin{abstract}	
	%\end{abstract}
	
	\part{Integrais duplas}
	
		\section{Invertendo os limites de integração - Aula 1}			
			\begin{enumerate}
				\item Exercício
				
				\begin{figure}[H]
					\centering
					\includegraphics[width=\textwidth]{v01_a01_e01.png}
					\caption{Integrais duplas - Aula 1 - Exercício I e II}
					\label{v01_a01_e01}
				\end{figure}
				
				$f(x) = x^2;\; g(x) = x^3$\newline
				$x = 0 \Rightarrow f(0) = g(0) \Rightarrow 0^2 = 0^3$\newline
				$x = 1 \Rightarrow f(1) = g(1) \Rightarrow 1^2 = 1^3$\newline
				
				$\integral_0^1 dx \integral_{g(x)}^{f(x)} dy = 
				\integral_0^1 dx \integral_{x^3}^{x^2} dy = 
				\integral_0^1 dx\, [y]_{x^3}^{x^2} =
				\integral_0^1 dx \left[x^2 - x^3\right] = 
				\integral_0^1 x^2\, dx - \integral_0^1 x^3\, dx = 
				\left[\dfrac{x^3}{3} - \dfrac{x^4}{4}\right]_0^1 = 
				\left[\dfrac{4x^3 - 3x^2}{12}\right]_0^1 = 
				\dfrac{1}{12}\left[4x^3 - 3x^2\right]_0^1 = 
				\dfrac{1}{12}\left[x^2\left(4x - 3\right)\right]_0^1 =
				\dfrac{1}{12}\left[1^2\left(4 \cdot 1 - 3\right) 
				\overstrike{- 0^2\left(4 \cdot 0 - 3\right)}\right] = 
				\dfrac{1}{12} = 0,08\overline{3}$\newline
								
				\item Exercício
				
				$f(x) = x^2 \Rightarrow f(y) = \sqrt{y};\; 
				g(x) = x^3 \Rightarrow g(y) = \sqrt[3]{y}$\newline
				$y = 0 \Rightarrow f(0) = g(0) \Rightarrow \sqrt{0} = \sqrt[3]{0}$\newline
				$y = 1 \Rightarrow f(1) = g(1) \Rightarrow \sqrt{1} = \sqrt[3]{1}$\newline
				
				$\integral_0^1 dy \integral_{f(y)}^{g(y)} dx = 
				\integral_0^1 dy \integral_{\sqrt{y}}^{\sqrt[3]{y}} dx = 
				\integral_0^1 dy\, [x]_{\sqrt{y}}^{\sqrt[3]{y}} = 
				\integral_0^1 dy \left[\sqrt[3]{y} - \sqrt{y}\right] = 
				\integral_0^1 \sqrt[3]{y}\, dy - \integral_0^1 \sqrt{y}\, dy = 
				\integral_0^1 y^{\frac{1}{3}}\, dy - 
				\integral_0^1 y^{\frac{1}{2}}\, dy = 
				\left[\dfrac{y^{\frac{4}{3}}}{\left(\dfrac{4}{3}\right)} - 
				\dfrac{y^{\frac{3}{2}}}{\left(\dfrac{3}{2}\right)}\right]_0^1 = 
				\left[\dfrac{3 \sqrt[3]{y^4}}{4} - \dfrac{2 \sqrt{y^3}}{3}\right]_0^1 = 
				\left[\dfrac{9 \sqrt[3]{y^4} - 8 \sqrt{y^3}}{12}\right]_0^1 = 
				\dfrac{1}{12}\left[9 \sqrt[3]{y^4} - 8 \sqrt{y^3}\right]_0^1 =\\ 
				\dfrac{1}{12}\left[\left(9 \sqrt[3]{1^4} - 8 \sqrt{1^3}\right) \overstrike{- 
				\left(9 \sqrt[3]{0^4} - 8 \sqrt{0^3}\right)}\right] = 
				\dfrac{1}{12}(9 - 8) = \dfrac{1}{12} = 0,08\overline{3}$
			\end{enumerate}
		
		\section{Determinação da região de integração - Aula 2}		
			\begin{enumerate}
				\item Exercício
				
				$R = \left\{(x, y) \in \mathbb{R}^2 \,|\, 0 \leq x \leq 2 \,,\, 
				0 \leq y \leq 6 \right\}$
				
				\begin{figure}[H]
					\centering
					\includegraphics[width=\textwidth]{v01_a02_e01.png}
					\caption{Integrais duplas - Aula 2 - Exercício I}
					\label{v01_a02_e01}
				\end{figure}
				
				$\integral_0^2 dx \integral_0^6 dy = 
				\integral_0^2 dx\, [y]_0^6 = 
				\integral_0^2 dx\, [6 - 0] = 
				6\integral_0^2 dx = 6[x]_0^2 = 6[2 - 0] = 6 \cdot 2 = 12 $\newline
				
				\item Exercício
				
				$R = \left\{(x, y) \in \mathbb{R} \,|\, 0 \leq x \leq 1 \,,\, 
				x \leq y \leq 2x \right\}$
									
				\begin{figure}[H]
					\centering
					\includegraphics[width=\textwidth]{v01_a02_e02.png}
					\caption{Integrais duplas - Aula 2 - Exercício II}
					\label{v01_a02_e02}
				\end{figure}
				
				$\integral_0^1 dx \integral_{x}^{2x} dy = 
				\integral_0^1 dx\, [y]_{x}^{2x} = \integral_0^1 dx\, [2x - x] =
				2\integral_0^1 x\, dx - \integral_0^1 x\, dx =
				\left[\overstrike{2}\dfrac{x^2}{\overstrike{2}} - 
				\dfrac{x^2}{2}\right]_0^1 = 
				\left[\dfrac{2x^2 - x^2}{2}\right]_0^1 = 
				\dfrac{1}{2}\left[x^2\right]_0^1 = 
				\dfrac{1}{2}\left[1^2 \overstrike{- 0^2}\right] = \dfrac{1}{2} = 0,5 $\newline
				
				\item Exercício
				
				$R = \left\{(x, y) \in \mathbb{R}^2 \,|\, 0 \leq y \leq 1 \,,\, 
				0 \leq x \leq \sqrt{1 - y^2} \right\}$
				
				$y = 0,\, y=1$\newline
				$x = 0,\, x = \sqrt{1 - y^2} \Rightarrow 
				x^2 = 1 - y^2 \Rightarrow x^2 - 1 = -y^2 \Rightarrow 
				y^2 = -x^2 + 1 \Rightarrow y = \sqrt{1 -x^2}$
								
				\begin{figure}[H]
					\centering
					\includegraphics[width=\textwidth]{v01_a02_e03.png}
					\caption{Integrais duplas - Aula 2 - Exercício III}
					\label{v01_a02_e03}
				\end{figure}
				
				$\integral_0^1 dy \integral_0^{f(y)} dx = 
				\integral_0^1 dy \integral_0^{\sqrt{1 - y^2}} dx = 
				\integral_0^1 dy\, [x]_0^{\sqrt{1 - y^2}} = 
				\integral_0^1 dy\, \left[\sqrt{1 - y^2} - 0\right] = 
				\integral_0^1 \sqrt{1 - y^2}\, dy = 
				\integral_0^1 \sqrt{1 - \sen^2(t)}\, \cos(t) dt = 
				\integral_0^1 \sqrt{\cos^2(t)}\, \cos(t) dt = 
				\integral_0^1 \cos(t)\cos(t) dt = 
				\integral_0^1 \cos^2(t) dt = 
				\integral_0^1 \dfrac{1 + \cos(2t)}{2} dt = 
				\dfrac{1}{2}\integral_0^1 \left[1 + \cos(2t)\right] dt = 
				\dfrac{1}{2}\integral_0^1 dt + 
				\dfrac{1}{2}\integral_0^1 \cos(2t) dt = 
				\dfrac{1}{2}\integral_0^1 dt + 
				\dfrac{1}{2}\integral_0^1 \cos(u) \dfrac{du}{2} = 
				\dfrac{1}{2}\integral_0^1 dt + 
				\dfrac{1}{4}\integral_0^1 \cos(u)\, du = 
				\left[\dfrac{1}{2}t + \dfrac{1}{4}\sen(u)\right]_0^1 = 
				\left[\dfrac{t}{2} + \dfrac{\sen(2t)}{4}\right]_0^1 = 
				\left[\dfrac{t}{2} + \dfrac{\overstrike{2}\sen(t)\cos(t)}{\overstrike{4}\,2}\right]_0^1 = 
				\left[\dfrac{t + \sen(t)\cos(t)}{2}\right]_0^1 = 
				\dfrac{1}{2}\left[\arcsen(y) + y\sqrt{1 - y^2}\right]_0^1 =\\ 
				\dfrac{1}{2}\left[\left(\arcsen(1) + \overstrike{1 \cdot\sqrt{1 - 1^2}}\right) - 
				\left(\arcsen(0) \overstrike{+ 0 \cdot \sqrt{1 - 0^2}}\right)\right] = \dfrac{1}{2}\left[\dfrac{\pi}{2} - 0\right] = \dfrac{\pi}{4} = 0,785$
				\newline\newline
				$y = \sen(t) \Rightarrow dy = \cos(t) dt$\newline
				$u = 2t \Rightarrow \dfrac{du}{2} = dt$\newline\newline
				$\sen(t) = \dfrac{co}{h} = \dfrac{y}{1} = y$\newline
				$h^2 = co^2 + ca^2 \Rightarrow 1 = y^2 + ca^2 \Rightarrow ca = \sqrt{1 - y^2}$\newline
				$\cos(t) = \dfrac{ca}{h} = \dfrac{\sqrt{1 - y^2}}{1} = \sqrt{1 - y^2}$\newline
				$y = \sen(t) \Rightarrow t = \arcsen(y)$
				
				\item Exercício
				
				$y = x^2 + 1 ,\, y = -x^2 - 1 ;\; x = 1 ,\, x = -1$\newline
				$R = \left\{(x, y) \in \mathbb{R} \,|\, -1 \leq x \leq 1 \,,\, 
				-x^2 - 1 \leq y \leq x^2 + 1 \right\}$
				
				\begin{figure}[H]
					\centering
					\includegraphics[width=\textwidth]{v01_a02_e04.png}
					\caption{Integrais duplas - Aula 2 - Exercício IV}
					\label{v01_a02_e04}
				\end{figure}
				
				$\integral_{-1}^1 dx \integral_{f(x)}^{g(x)} dy = 
				\integral_{-1}^1 dx \integral_{-x^2 - 1}^{x^2 + 1} dy = 
				\integral_{-1}^1 dx\, [y]_{-x^2 - 1}^{x^2 + 1} = 
				\integral_{-1}^1 dx\, \left[x^2 + 1 - \left(-x^2 - 1\right)\right] = 
				\integral_{-1}^1 dx\, \left[x^2 + 1 + x^2 + 1\right] = 
				\integral_{-1}^1 dx\, \left[2x^2 + 2\right] = 
				2\integral_{-1}^1 x^2\, dx + 2\integral_{-1}^1 dx = 
				\left[2\dfrac{x^3}{3} +  2x\right]_{-1}^1 = 
				\left[2\left(\dfrac{x^3 + 3x}{3}\right)\right]_{-1}^1 = 
				\dfrac{2}{3}\left[x\left(x^2 + 3\right)\right]_{-1}^1 = \\ 
				\dfrac{2}{3}\left[1 \cdot \left(1^2 + 3\right) - (-1)\left((-1)^2 + 3\right)\right] =
				\dfrac{2}{3}(4 + 4) = \dfrac{2}{3}8 = \dfrac{16}{3} = 5,\overline{3}$
				
				\item Exercício
				
				$R = \left\{(x, y) \in \mathbb{R} \,|\, 0 \leq y \leq 2 \,,\, 
				-y \leq x \leq y \right\}$
				
				\begin{figure}[H]
					\centering
					\includegraphics[width=\textwidth]{v01_a02_e05.png}
					\caption{Integrais duplas - Aula 2 - Exercício V}
					\label{v01_a02_e05}
				\end{figure}
				
				$\integral_0^2 dy \integral_{f(y)}^{g(y)} dx = \integral_0^2 dy \integral_{-y}^y dx =
				\integral_0^2 dy\, [x]_{-y}^y = \integral_0^2 dy\, [y - (-y)] = 
				\integral_0^2 dy\, [2y] = 2\integral_0^2 y\, dy = \left[\overstrike{2}\frac{y^2}{\overstrike{2}}\right]_0^2 = 2^2 - 0^2 = 4$
				
			\end{enumerate}
		
		%\section{Cálculo de volume - Aula 3}
		
		%\section{Invertendo a ordem de integração - Aula 4}
		
		%\section{Cálculo de integrais duplas ou iteradas}
			%\subsection{Aula 5}
		
			%\subsection{Aula 6}
		
			%\subsection{Aula 7}
		
		%\section{Cálculo de área - Aula 8}
		
		%\section{Cálculo de volume}
			%\subsection{Aula 9}
		
			%\subsection{Aula 10}		
	
		%\section{Coordenadas polares}		
			%\subsection{Aula 1}
			
			%\subsection{Aula 2}
			
			%\subsection{Aula 3}
			
	%\part{Integrais triplas}
	
		%\section{Introdução - Aula 1}
		
		%\section{Cálculo de integrais triplas - Aula 2}
		
		%\section{Cálculo do volume de um sólido - Aula 3}
		
		%\section{Esboço de um sólido - Aula 4}
	
		%\section{Coordenadas esféricas}
			%\subsection{Aula 1}
			
			%\subsection{Aula 2}
			
			%\subsection{Aula 3}
			
			%\subsection{Aula 4}
			
			%\subsection[Aula 5]{Cálculo de massa com coordenadas esféricas - Aula 5}
			
			%\subsection{Aula 6}
			
			%\subsection{Aula 7}
	
		%\section{Coordenadas cilíndricas}
			%\subsection{Aula 1}
			
			%\subsection{Aula 2}
			
			%\subsection{Aula 3}
			
			%\subsection{Aula 4}
			
			%\subsection{Aula 5}
			
			%\subsection{Aula 6}
			
			%\subsection{Aula 7}
			
			%\subsection{Aula 8}
\end{document}
